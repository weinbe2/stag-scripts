
\centerline{\textbf{Goldstone Boson Pion, Wall Source}}

This is the $\gamma_5 \otimes \gamma_5$ state, no parity partner, measured with a wall source.

{\small{Used :ACT_MEAS: measurements separated by :FREQ_MEAS: trajectories (:MISS_MEAS: missing). Blocksize of :PS_MEASPERBLOCK: measurements. Thermalization cut of trajectory :THERM:. Measurements performed up to trajectory :LAST_MEAS:. Prefered $t_{min}$ is :PS_TMIN:, consistent $t_{max}$ of :PS_TMAX: was used for all fits.}}

\begin{table}[ht!]
\centering
\scriptsize
\begin{tabular}{|c|c|c|c|}
\hline
 $t_{min}$ & $c_{\pi, GB}$ &  $M_{\pi, GB}$ & $p$-value \\
\hline
\input{/projectnb/qcd/Staggered/:ENSEMBLE:/spectrum2/tex/table_new.ps}
\end{tabular}
\caption{Fit values for the goldstone boson pion amplitude and mass, defined by the fit form in Section~\ref{sec:coshcorr}.}
\end{table}

\begin{multicols}{2}
\begin{figure}[H]
\centering
\includegraphics[width=2.5in]{/projectnb/qcd/Staggered/:ENSEMBLE:/spectrum2/effmass/plots/effmass-ps1.pdf}
\caption{One-state effective mass plot for the Goldstone boson pion (Cosh) from a wall source, defined in Section~\ref{sec:threepteffmass}.}
\end{figure}
\columnbreak
\begin{figure}[H]
\centering
\includegraphics[width=2.5in]{/projectnb/qcd/Staggered/:ENSEMBLE:/spectrum2/fits/plots/fit-ps1.pdf}
\caption{Non-linear fit mass of the Goldstone boson pion (Cosh) from a wall source, defined by the fit form in Section~\ref{sec:coshcorr}.}
\end{figure}
\end{multicols}

\begin{figure}[H]
\centering
\includegraphics[width=2.5in]{/projectnb/qcd/Staggered/:ENSEMBLE:/spectrum2/sum/plots/sum-ps.pdf}
\caption{Log correlator plot for the Goldstone boson pion from a wall source, overlaid with preferred fit curve, defined by the fit form in Section~\ref{sec:coshcorr}.}
\end{figure}

\clearpage

\centerline{\textbf{Goldstone Boson Pion, Random Wall Source}}

This is the $\gamma_5 \otimes \gamma_5$ state, no parity partner, measured with a random wall source.

{\small{Used :ACT_MEAS: measurements separated by :FREQ_MEAS: trajectories (:MISS_MEAS: missing). Blocksize of :PS2_MEASPERBLOCK: measurements. Thermalization cut of trajectory :THERM:. Measurements performed up to trajectory :LAST_MEAS:. Prefered $t_{min}$ is :PS2_TMIN:, consistent $t_{max}$ of :PS2_TMAX: was used for all fits.}}

\begin{table}[ht!]
\centering
\scriptsize
\begin{tabular}{|c|c|c|c|c|}
\hline
 $t_{min}$ & $c_{\pi, GB}$ &  $M_{\pi, GB}$ & $f_{\pi,GB}$ & $p$-value \\
\hline
\input{/projectnb/qcd/Staggered/:ENSEMBLE:/spectrum2/tex/table_new.ps2}
\end{tabular}
\caption{Fit values for the goldstone boson pion amplitude and mass, defined by the fit form in Section~\ref{sec:coshcorr}.}
\end{table}

\begin{multicols}{2}
\begin{figure}[H]
\centering
\includegraphics[width=2.5in]{/projectnb/qcd/Staggered/:ENSEMBLE:/spectrum2/effmass/plots/effmass-ps21.pdf}
\caption{One-state effective mass plot for the Goldstone boson pion (Cosh) from a random wall source (point source), defined in Section~\ref{sec:threepteffmass}.}
\end{figure}
\columnbreak
\begin{figure}[H]
\centering
\includegraphics[width=2.5in]{/projectnb/qcd/Staggered/:ENSEMBLE:/spectrum2/fits/plots/fit-ps21.pdf}
\caption{Non-linear fit mass of the Goldstone boson pion (Cosh) from a random wall source, defined by the fit form in Section~\ref{sec:coshcorr}.}
\end{figure}
\end{multicols}

\begin{multicols}{2}
\begin{figure}[H]
\centering
\includegraphics[width=2.5in]{/projectnb/qcd/Staggered/:ENSEMBLE:/spectrum2/fits/plots/fit-ps22.pdf}
\caption{Non-linear fit value of $f_\pi$ for the Goldstone boson pion from a random wall source.}
\end{figure}
\columnbreak
\begin{figure}[H]
\centering
\includegraphics[width=2.5in]{/projectnb/qcd/Staggered/:ENSEMBLE:/spectrum2/sum/plots/sum-ps2.pdf}
\caption{Log correlator plot for the Goldstone boson pion from a random wall source, overlaid with preferred fit curve, defined by the fit form in Section~\ref{sec:coshcorr}.}
\end{figure}
\end{multicols}


\clearpage

\centerline{\textbf{Rho and Partner Axial, Wall Source}}

This is the $\gamma_3 \gamma_4 \otimes \gamma_1 \gamma_5$ rho, parity partner $\gamma_3 \gamma_5 \otimes \gamma_1 \gamma_4$ axial vector, measured with a wall source. 

{\small{Used :ACT_MEAS: measurements separated by :FREQ_MEAS: trajectories (:MISS_MEAS: missing). Blocksize of :RI5_MEASPERBLOCK: measurements. Thermalization cut of trajectory :THERM:. Measurements performed up to trajectory :LAST_MEAS:. Prefered $t_{min}$ is :RI5_TMIN:, consistent $t_{max}$ of :RI5_TMAX: was used for all fits.}}


\begin{table}[ht!]
\centering
\scriptsize
\begin{tabular}{|c|c|c|c|c|c|}
\hline
 $t_{min}$ & $c_{\rho}$ &  $M_{\rho}$ & $c_{axial}$ & $M_{axial}$ & $p$-value \\
\hline
\input{/projectnb/qcd/Staggered/:ENSEMBLE:/spectrum2/tex/table_new.ri5}
\end{tabular}
\caption{Fit values for a rho meson and its partner axial vector, defined by the fit form in Section~\ref{sec:coshoscilcorr}.}
\end{table}

\begin{multicols}{2}
\begin{figure}[H]
\centering
\includegraphics[width=2.5in]{/projectnb/qcd/Staggered/:ENSEMBLE:/spectrum2/fits/plots/fit-ri51.pdf}
\caption{Non-linear fit mass of the rho meson (Cosh) from a wall source, defined by the fit form in Section~\ref{sec:coshoscilcorr}.}
\end{figure}
\columnbreak
\begin{figure}[H]
\centering
\includegraphics[width=2.5in]{/projectnb/qcd/Staggered/:ENSEMBLE:/spectrum2/fits/plots/fit-ri52.pdf}
\caption{Non-linear fit mass of the axial vector meson (Oscil) from a wall source, defined by the fit form in Section~\ref{sec:coshoscilcorr}.}
\end{figure}
\end{multicols}


\begin{figure}[H]
\centering
\includegraphics[width=2.5in]{/projectnb/qcd/Staggered/:ENSEMBLE:/spectrum2/sum/plots/sum-ri5.pdf}
\caption{Log correlator plot for a rho meson and its partner axial vector from a wall source, overlaid with preferred fit curve, defined by the fit form in Section~\ref{sec:coshoscilcorr}.}
\end{figure}

\clearpage

\centerline{\textbf{Rho and Partner Tensor, Wall Source}}

This is the $\gamma_3 \otimes \gamma_2 \gamma_3$ rho, parity partner $\gamma_1 \gamma_2 \otimes \gamma_1$ tensor, measured with a wall source.

{\small{Used :ACT_MEAS: measurements separated by :FREQ_MEAS: trajectories (:MISS_MEAS: missing). Blocksize of :RIJ_MEASPERBLOCK: measurements. Thermalization cut of trajectory :THERM:. Measurements performed up to trajectory :LAST_MEAS:. Prefered $t_{min}$ is :RIJ_TMIN:, consistent $t_{max}$ of :RIJ_TMAX: was used for all fits.}}


\begin{table}[ht!]
\centering
\scriptsize
\begin{tabular}{|c|c|c|c|c|c|}
\hline
 $t_{min}$ & $c_{\rho}$ &  $M_{\rho}$ & $c_{tensor}$ & $M_{tensor}$ & $p$-value \\
\hline
\input{/projectnb/qcd/Staggered/:ENSEMBLE:/spectrum2/tex/table_new.rij}
\end{tabular}
\caption{Fit values for a rho meson and its partner axial vector, defined by the fit form in Section~\ref{sec:coshoscilcorr}.}
\end{table}

\begin{multicols}{2}
\begin{figure}[H]
\centering
\includegraphics[width=2.5in]{/projectnb/qcd/Staggered/:ENSEMBLE:/spectrum2/fits/plots/fit-rij1.pdf}
\caption{Non-linear fit mass of the rho meson (Cosh) from a wall source, defined by the fit form in Section~\ref{sec:coshoscilcorr}.}
\end{figure}
\columnbreak
\begin{figure}[H]
\centering
\includegraphics[width=2.5in]{/projectnb/qcd/Staggered/:ENSEMBLE:/spectrum2/fits/plots/fit-rij2.pdf}
\caption{Non-linear fit mass of the tensor meson (Oscil) from a wall source, defined by the fit form in Section~\ref{sec:coshoscilcorr}.}
\end{figure}
\end{multicols}

\begin{figure}[H]
\centering
\includegraphics[width=2.5in]{/projectnb/qcd/Staggered/:ENSEMBLE:/spectrum2/sum/plots/sum-ri5.pdf}
\caption{Log correlator plot for a rho meson and its partner tensor from a wall source, overlaid with preferred fit curve, defined by the fit form in Section~\ref{sec:coshoscilcorr}.}
\end{figure}


\clearpage

\centerline{\textbf{Connected Scalar and Partner Pion, Wall Source}}

This is the $\gamma_4 \gamma_5 \otimes \gamma_4 \gamma_5$ pion, parity partner $1 \otimes 1$ connected taste singlet scalar, measured with a wall source. 

{\small{Used :ACT_MEAS: measurements separated by :FREQ_MEAS: trajectories (:MISS_MEAS: missing). Blocksize of :SC_MEASPERBLOCK: measurements. Thermalization cut of trajectory :THERM:. Measurements performed up to trajectory :LAST_MEAS:. Prefered $t_{min}$ is :SC_TMIN:, consistent $t_{max}$ of :SC_TMAX: was used for all fits.}}

\begin{table}[ht!]
\centering
\scriptsize
\begin{tabular}{|c|c|c|c|c|c|}
\hline
 $t_{min}$ & $c_{\pi_{sc}}$ &  $M_{\pi_{sc}}$ & $c_{a_0}$ & $M_{a_0}$ & $p$-value \\
\hline
\input{/projectnb/qcd/Staggered/:ENSEMBLE:/spectrum2/tex/table_new.sc}
\end{tabular}
\caption{Fit values for the connected scalar and partner pion, measured from a wall source, defined by the fit form in Section~\ref{sec:coshoscilcorr}.}
\end{table}


\begin{multicols}{2}
\begin{figure}[H]
\centering
\includegraphics[width=2.5in]{/projectnb/qcd/Staggered/:ENSEMBLE:/spectrum2/fits/plots/fit-sc1.pdf}
\caption{Non-linear fit mass of the $\gamma_4$ pion (Cosh) from a wall source, defined by the fit form in Section~\ref{sec:coshoscilcorr}.}
\end{figure}
\columnbreak
\begin{figure}[H]
\centering
\includegraphics[width=2.5in]{/projectnb/qcd/Staggered/:ENSEMBLE:/spectrum2/fits/plots/fit-sc2.pdf}
\caption{Non-linear fit mass of the connected scalar meson (Oscil) from a wall source, defined by the fit form in Section~\ref{sec:coshoscilcorr}.}
\end{figure}
\end{multicols}

\begin{figure}[H]
\centering
\includegraphics[width=2.5in]{/projectnb/qcd/Staggered/:ENSEMBLE:/spectrum2/sum/plots/sum-sc.pdf}
\caption{Log correlator plot for the connected scalar (Oscil) and its partner pion (Cosh) from a wall source, overlaid with preferred fit curve, defined by the fit form in Section~\ref{sec:coshoscilcorr}.}
\end{figure}

\clearpage


\begin{comment}
\centerline{\textbf{Connected Scalar and Partner Pion, Stochastic Source}}

This is the $1 \otimes 1$ connected taste singlet scalar, parity partner $\gamma_4 \gamma_5 \otimes \gamma_4 \gamma_5$ pion, measured with a stochastic source. Please note that this is switched from the wall source!

\begin{table}[ht!]
\centering
\begin{tabular}{|c|c|c|c|c|c|}
\hline
 $t_{min}$ & $c_{a_0}$ &  $M_{a_0}$ & $c_{\pi_{sc}}$ & $M_{\pi_{sc}}$ & $p$-value \\
\hline
\input{../../AllCode/2014-08-18CalculationSuite2/ensembles/:ENSEMBLE:/tex/table_new.sc_stoch}
\end{tabular}
\caption{Fit values for the connected scalar and partner pion, measured from a random wall source. This correlator is used in the connected subtraction for the sigma.}
\end{table}


\begin{figure}[H]
\centering
\includegraphics[width=3.75in]{../../AllCode/2014-08-18CalculationSuite2/ensembles/:ENSEMBLE:/effmass/plots/effmass-sc_stoch.pdf}
\caption{Two-state effective mass plot for the connected scalar (Cosh) and its partner pion (Oscil) from a random wall source (point source).}
\end{figure}

\begin{figure}[H]
\centering
\includegraphics[width=3.75in]{../../AllCode/2014-08-18CalculationSuite2/ensembles/:ENSEMBLE:/sum/plots/sum-sc_stoch.pdf}
\caption{Arcsinh correlator plot for the connected scalar and its partner pion from a random wall source (point source), overlaid with preferred fit curve.}
\end{figure}

\clearpage

\centerline{\textbf{Sigma, No Analytic Subtraction}}

This is the $1 \otimes 1$ connected taste singlet isosinglet scalar, parity partner $\gamma_4 \gamma_5 \otimes \gamma_4 \gamma_5$ isosinglet pion, measured with a stochastic source. We measure this directly from $N_f D - C$.

\begin{table}[ht!]
\centering
\begin{tabular}{|c|c|c|c|c|c|c|}
\hline
 $t_{min}$ & $c_{\sigma}$ &  $M_{\sigma}$ & $c_{\pi_{\bar{sc}}}$ & $M_{\pi_{\bar{sc}}}$ & $p$-value \\
\hline
\input{../../AllCode/2014-08-18CalculationSuite2/ensembles/:ENSEMBLE:/tex/table_new.sg_111}
\end{tabular}
\end{table}

\begin{figure}[H]
\centering
\includegraphics[width=3.75in]{../../AllCode/2014-08-18CalculationSuite2/ensembles/:ENSEMBLE:/effmass/plots/effmass-sg_111.pdf}
\caption{Two-state effective mass plot for the sigma meson (Cosh) and its partner pion (Oscil) from stochastic sources.}
\end{figure}

\begin{figure}[H]
\centering
\includegraphics[width=3.75in]{../../AllCode/2014-08-18CalculationSuite2/ensembles/:ENSEMBLE:/sum/plots/sum-sg_111.pdf}
\caption{Arcsinh correlator plot for the sigma meson and its partner pion from stochastic sources, overlaid with preferred fit curve.}
\end{figure}

\clearpage

\centerline{\textbf{Sigma, With Analytic Subtraction}}

This is the $1 \otimes 1$ connected taste singlet isosinglet scalar, parity partner $\gamma_4 \gamma_5 \otimes \gamma_4 \gamma_5$ isosinglet pion, measured with a stochastic source. We measure this using an analytic subtraction of the connected piece.

\begin{table}[ht!]
\centering
\begin{tabular}{|c|c|c|c|c|c|}
\hline
 $t_{min}$ & $c_{\sigma}$ &  $M_{\sigma}$ & $c_{\pi_{\bar{sc}}}$ & $M_{\pi_{\bar{sc}}}$ & $p$-value \\
\hline
\input{../../AllCode/2014-08-18CalculationSuite2/ensembles/:ENSEMBLE:/tex/table_new.sg_211}
\end{tabular}
\end{table}

\begin{figure}[H]
\centering
\includegraphics[width=3.75in]{../../AllCode/2014-08-18CalculationSuite2/ensembles/:ENSEMBLE:/effmass/plots/effmass-sg_211.pdf}
\caption{Two-state effective mass plot for the sigma meson (Cosh) and its partner pion (Oscil) from stochastic sources.}
\end{figure}

\begin{figure}[H]
\centering
\includegraphics[width=3.75in]{../../AllCode/2014-08-18CalculationSuite2/ensembles/:ENSEMBLE:/sum/plots/sum-sg_211.pdf}
\caption{Arcsinh correlator plot for the sigma meson and its partner pion from stochastic sources, overlaid with preferred fit curve.}
\end{figure}

\clearpage

\end{comment}
