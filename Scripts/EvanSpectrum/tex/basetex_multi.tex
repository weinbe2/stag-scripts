
\subsubsection{Goldstone Boson Pion, Wall Source}

Ensemble: :ENSEMBLE:

This is the $\gamma_5 \otimes \gamma_5$ state, no parity partner, measured with a wall source.

{\small{Used :ACT_MEAS: measurements separated by :FREQ_MEAS: trajectories (:MISS_MEAS: missing). Blocksize of :PS_MEASPERBLOCK: measurements. Thermalization cut of trajectory :THERM:. Measurements performed up to trajectory :LAST_MEAS:. Prefered $t_{min}$ is :PS_TMIN:, consistent $t_{max}$ of :PS_TMAX: was used for all fits.}}

\begin{comment}
\begin{table}[ht!]
\centering
\scriptsize
\begin{tabular}{|c|c|c|c|}
\hline
 $t_{min}$ & $c_{\pi, GB}$ &  $M_{\pi, GB}$ & $p$-value \\
\hline
\input{/projectnb/qcd/Staggered/:ENSEMBLE:/spectrum2/tex/table_new.ps}
\end{tabular}
\caption{Fit values for the goldstone boson pion amplitude and mass, defined by the fit form in Section~\ref{sec:coshcorr}.}
\end{table}
\end{comment}

\begin{multicols}{2}
\begin{figure}[H]
\centering
\includegraphics[width=2.5in]{/projectnb/qcd/Staggered/:ENSEMBLE:/spectrum2/multifits/plots/multifits-ps.pdf}
\caption{Non-linear fit mass for the Goldstone boson pion from a wall source.}
\end{figure}
\columnbreak
\begin{figure}[H]
\centering
\includegraphics[width=2.5in]{/projectnb/qcd/Staggered/:ENSEMBLE:/spectrum2/multifits/plots/multifits-ps-pv.pdf}
\caption{$p$ value for a non-linear fit of the Goldstone boson pion from a wall source.}
\end{figure}
\end{multicols}

\begin{figure}[H]
\centering
\includegraphics[width=2.5in]{/projectnb/qcd/Staggered/:ENSEMBLE:/spectrum2/multifits/plots/multifits-ps-dir.pdf}
\caption{Zoom in on ground state mass of Goldstone boson pion from a wall source.}
\end{figure}

\begin{figure}[H]
\centering
\includegraphics[width=2.5in]{/projectnb/qcd/Staggered/:ENSEMBLE:/spectrum2/multifits/plots/multifits-ps-dir.pdf}
\caption{I should be a correlator plot!}
\end{figure}

\clearpage

\subsubsection{Goldstone Boson Pion, Random Wall Source}

Ensemble: :ENSEMBLE:

This is the $\gamma_5 \otimes \gamma_5$ state, no parity partner, measured with a random wall source.

{\small{Used :ACT_MEAS: measurements separated by :FREQ_MEAS: trajectories (:MISS_MEAS: missing). Blocksize of :PS2_MEASPERBLOCK: measurements. Thermalization cut of trajectory :THERM:. Measurements performed up to trajectory :LAST_MEAS:. Prefered $t_{min}$ is :PS2_TMIN:, consistent $t_{max}$ of :PS2_TMAX: was used for all fits.}}

\begin{multicols}{2}
\begin{figure}[H]
\centering
\includegraphics[width=2.5in]{/projectnb/qcd/Staggered/:ENSEMBLE:/spectrum2/multifits/plots/multifits-ps2.pdf}
\caption{Non-linear fit mass for the Goldstone boson pion from a point source.}
\end{figure}
\columnbreak
\begin{figure}[H]
\centering
\includegraphics[width=2.5in]{/projectnb/qcd/Staggered/:ENSEMBLE:/spectrum2/multifits/plots/multifits-ps2-pv.pdf}
\caption{$p$ value for a non-linear fit of the Goldstone boson pion from a point source.}
\end{figure}
\end{multicols}

\begin{multicols}{2}
\begin{figure}[H]
\centering
\includegraphics[width=2.5in]{/projectnb/qcd/Staggered/:ENSEMBLE:/spectrum2/multifits/plots/multifits-ps2-dir.pdf}
\caption{Zoom in on ground state mass of Goldstone boson pion from a point source.}
\end{figure}
\columnbreak
\begin{figure}[H]
\centering
\includegraphics[width=2.5in]{/projectnb/qcd/Staggered/:ENSEMBLE:/spectrum2/multifits/plots/multifits-ps2-osc.pdf}
\caption{Zoom in on $F_\pi$ from Goldstone boson pion from a point source.}
\end{figure}
\end{multicols}

\begin{figure}[H]
\centering
\includegraphics[width=2.5in]{/projectnb/qcd/Staggered/:ENSEMBLE:/spectrum2/multifits/plots/multifits-ps2-dir.pdf}
\caption{I should be a correlator plot!}
\end{figure}

\clearpage

:TASTE_YES_BEGIN:
\subsubsection{Taste Split i5 Pion w/ No Partner}

Ensemble: :ENSEMBLE:

This is the $\gamma_5 \otimes \gamma_5 \gamma_3$ pion, without parity partner, measured with a wall source. 

{\small{Used :ACT_MEAS: measurements separated by :FREQ_MEAS: trajectories (:MISS_MEAS: missing). Blocksize of :I5_MEASPERBLOCK: measurements. Thermalization cut of trajectory :THERM:. Measurements performed up to trajectory :LAST_MEAS:. Prefered $t_{min}$ is :I5_TMIN:, consistent $t_{max}$ of :I5_TMAX: was used for all fits.}}

\begin{multicols}{2}
\begin{figure}[H]
\centering
\includegraphics[width=2.5in]{/projectnb/qcd/Staggered/:ENSEMBLE:/spectrum2/multifits/plots/multifits-i5.pdf}
\caption{Non-linear fit mass for the i5 taste split pion from a point source.}
\end{figure}
\columnbreak
\begin{figure}[H]
\centering
\includegraphics[width=2.5in]{/projectnb/qcd/Staggered/:ENSEMBLE:/spectrum2/multifits/plots/multifits-i5-pv.pdf}
\caption{$p$ value for a non-linear fit of the i5 taste split pion from a point source.}
\end{figure}
\end{multicols}


\begin{figure}[H]
\centering
\includegraphics[width=2.5in]{/projectnb/qcd/Staggered/:ENSEMBLE:/spectrum2/multifits/plots/multifits-i5-dir.pdf}
\caption{Zoom in on the ground state mass of the i5 pion from a wall source.}
\end{figure}

\begin{figure}[H]
\centering
\includegraphics[width=2.5in]{/projectnb/qcd/Staggered/:ENSEMBLE:/spectrum2/multifits/plots/multifits-i5-dir.pdf}
\caption{I should be a correlator plot!}
\end{figure}

\begin{figure}[H]
\centering
\includegraphics[width=2.5in]{/projectnb/qcd/Staggered/:ENSEMBLE:/spectrum2/multifits/plots/multifits-i5-dir.pdf}
\caption{I should be a correlator plot!}
\end{figure}

\clearpage
:TASTE_YES_END:

\subsubsection{Connected Scalar and Partner sc Pion, Wall Source}

Ensemble: :ENSEMBLE:

This is the $\gamma_4 \gamma_5 \otimes \gamma_4 \gamma_5$ pion, parity partner $1 \otimes 1$ connected taste singlet scalar, measured with a wall source. 

{\small{Used :ACT_MEAS: measurements separated by :FREQ_MEAS: trajectories (:MISS_MEAS: missing). Blocksize of :SC_MEASPERBLOCK: measurements. Thermalization cut of trajectory :THERM:. Measurements performed up to trajectory :LAST_MEAS:. Prefered $t_{min}$ is :SC_TMIN:, consistent $t_{max}$ of :SC_TMAX: was used for all fits.}}

\begin{multicols}{2}
\begin{figure}[H]
\centering
\includegraphics[width=2.5in]{/projectnb/qcd/Staggered/:ENSEMBLE:/spectrum2/multifits/plots/multifits-sc.pdf}
\caption{Non-linear fit mass for the sc taste split pion with partner $a_0$ from a wall source.}
\end{figure}
\columnbreak
\begin{figure}[H]
\centering
\includegraphics[width=2.5in]{/projectnb/qcd/Staggered/:ENSEMBLE:/spectrum2/multifits/plots/multifits-sc-pv.pdf}
\caption{$p$ value for a non-linear fit of the sc taste split pion with partner $a_0$ from a wall source.}
\end{figure}
\end{multicols}

\begin{multicols}{2}
\begin{figure}[H]
\centering
\includegraphics[width=2.5in]{/projectnb/qcd/Staggered/:ENSEMBLE:/spectrum2/multifits/plots/multifits-sc-dir.pdf}
\caption{Zoom in on ground state mass of the sc taste split pion from a wall source.}
\end{figure}
\columnbreak
\begin{figure}[H]
\centering
\includegraphics[width=2.5in]{/projectnb/qcd/Staggered/:ENSEMBLE:/spectrum2/multifits/plots/multifits-sc-osc.pdf}
\caption{Zoom in on ground state mass of the $a_0$ from a wall source.}
\end{figure}
\end{multicols}

\begin{figure}[H]
\centering
\includegraphics[width=2.5in]{/projectnb/qcd/Staggered/:ENSEMBLE:/spectrum2/multifits/plots/multifits-sc-dir.pdf}
\caption{I should be a correlator plot!}
\end{figure}

\clearpage

:STOCH_YES_BEGIN:
\subsubsection{Connected Scalar and Partner sc Pion, Stochastic Source}

Ensemble: :ENSEMBLE:

This is the $1 \otimes 1$ connected taste singlet scalar, parity partner $\gamma_4 \gamma_5 \otimes \gamma_4 \gamma_5$ pion, measured with a stochastic source. Please note that this is switched from the wall source!

\begin{multicols}{2}
\begin{figure}[H]
\centering
\includegraphics[width=2.5in]{/projectnb/qcd/Staggered/:ENSEMBLE:/spectrum2/multifits/plots/multifits-sc_stoch.pdf}
\caption{Non-linear fit mass for the $a_0$ with partner sc taste split pion from a point source.}
\end{figure}
\columnbreak
\begin{figure}[H]
\centering
\includegraphics[width=2.5in]{/projectnb/qcd/Staggered/:ENSEMBLE:/spectrum2/multifits/plots/multifits-sc_stoch-pv.pdf}
\caption{$p$ value for a non-linear fit of the $a_0$ with partner sc taste split pion from a point source.}
\end{figure}
\end{multicols}

\begin{multicols}{2}
\begin{figure}[H]
\centering
\includegraphics[width=2.5in]{/projectnb/qcd/Staggered/:ENSEMBLE:/spectrum2/multifits/plots/multifits-sc_stoch-dir.pdf}
\caption{Zoom in on ground state mass of the $a_0$ from a point source.}
\end{figure}
\columnbreak
\begin{figure}[H]
\centering
\includegraphics[width=2.5in]{/projectnb/qcd/Staggered/:ENSEMBLE:/spectrum2/multifits/plots/multifits-sc_stoch-osc.pdf}
\caption{Zoom in on ground state mass of the sc taste split pion from a point source.}
\end{figure}
\end{multicols}

\begin{figure}[H]
\centering
\includegraphics[width=2.5in]{/projectnb/qcd/Staggered/:ENSEMBLE:/spectrum2/multifits/plots/multifits-sc_stoch-dir.pdf}
\caption{I should be a correlator plot!}
\end{figure}

\clearpage
:STOCH_YES_END:

:TASTE_YES_BEGIN:
\subsubsection{Taste Split Connected Scalar and Partner ij Pion, Wall Source}

Ensemble: :ENSEMBLE:

This is the $\gamma_4 \gamma_5 \otimes \gamma_1 \gamma_2$ pion, parity partner $1 \otimes \gamma_3$ connected taste singlet scalar, measured with a wall source. 

{\small{Used :ACT_MEAS: measurements separated by :FREQ_MEAS: trajectories (:MISS_MEAS: missing). Blocksize of :IJ_MEASPERBLOCK: measurements. Thermalization cut of trajectory :THERM:. Measurements performed up to trajectory :LAST_MEAS:. Prefered $t_{min}$ is :IJ_TMIN:, consistent $t_{max}$ of :IJ_TMAX: was used for all fits.}}

\begin{multicols}{2}
\begin{figure}[H]
\centering
\includegraphics[width=2.5in]{/projectnb/qcd/Staggered/:ENSEMBLE:/spectrum2/multifits/plots/multifits-ij.pdf}
\caption{Non-linear fit mass for the ij taste split pion with partner taste split $a_0$ from a wall source.}
\end{figure}
\columnbreak
\begin{figure}[H]
\centering
\includegraphics[width=2.5in]{/projectnb/qcd/Staggered/:ENSEMBLE:/spectrum2/multifits/plots/multifits-ij-pv.pdf}
\caption{$p$ value for a non-linear fit of the ij taste split pion with partner taste split $a_0$ from a wall source.}
\end{figure}
\end{multicols}

\begin{multicols}{2}
\begin{figure}[H]
\centering
\includegraphics[width=2.5in]{/projectnb/qcd/Staggered/:ENSEMBLE:/spectrum2/multifits/plots/multifits-ij-dir.pdf}
\caption{Zoom in on ground state mass of the ij taste split pion from a wall source.}
\end{figure}
\columnbreak
\begin{figure}[H]
\centering
\includegraphics[width=2.5in]{/projectnb/qcd/Staggered/:ENSEMBLE:/spectrum2/multifits/plots/multifits-ij-osc.pdf}
\caption{Zoom in on ground state mass of the taste split $a_0$ from a wall source.}
\end{figure}
\end{multicols}

\begin{figure}[H]
\centering
\includegraphics[width=2.5in]{/projectnb/qcd/Staggered/:ENSEMBLE:/spectrum2/multifits/plots/multifits-ij-dir.pdf}
\caption{I should be a correlator plot!}
\end{figure}

\clearpage
:TASTE_YES_END:

\subsubsection{Rho and Partner Axial, Wall Source}

Ensemble: :ENSEMBLE:

This is the $\gamma_3 \gamma_4 \otimes \gamma_1 \gamma_5$ rho, parity partner $\gamma_3 \gamma_5 \otimes \gamma_1 \gamma_4$ axial vector, measured with a wall source. 

{\small{Used :ACT_MEAS: measurements separated by :FREQ_MEAS: trajectories (:MISS_MEAS: missing). Blocksize of :RI5_MEASPERBLOCK: measurements. Thermalization cut of trajectory :THERM:. Measurements performed up to trajectory :LAST_MEAS:. Prefered $t_{min}$ is :RI5_TMIN:, consistent $t_{max}$ of :RI5_TMAX: was used for all fits.}}


\begin{multicols}{2}
\begin{figure}[H]
\centering
\includegraphics[width=2.5in]{/projectnb/qcd/Staggered/:ENSEMBLE:/spectrum2/multifits/plots/multifits-ri5.pdf}
\caption{Non-linear fit mass for the vector with partner $a_1$ axial from a wall source.}
\end{figure}
\columnbreak
\begin{figure}[H]
\centering
\includegraphics[width=2.5in]{/projectnb/qcd/Staggered/:ENSEMBLE:/spectrum2/multifits/plots/multifits-ri5-pv.pdf}
\caption{$p$ value for a non-linear fit of the vector with partner $a_1$ axial from a wall source.}
\end{figure}
\end{multicols}

\begin{multicols}{2}
\begin{figure}[H]
\centering
\includegraphics[width=2.5in]{/projectnb/qcd/Staggered/:ENSEMBLE:/spectrum2/multifits/plots/multifits-ri5-dir.pdf}
\caption{Zoom in on ground state mass of the vector from a wall source.}
\end{figure}
\columnbreak
\begin{figure}[H]
\centering
\includegraphics[width=2.5in]{/projectnb/qcd/Staggered/:ENSEMBLE:/spectrum2/multifits/plots/multifits-ri5-osc.pdf}
\caption{Zoom in on ground state mass of the $a_1$ axial from a wall source.}
\end{figure}
\end{multicols}

\begin{figure}[H]
\centering
\includegraphics[width=2.5in]{/projectnb/qcd/Staggered/:ENSEMBLE:/spectrum2/multifits/plots/multifits-ri5-dir.pdf}
\caption{I should be a correlator plot!}
\end{figure}

\clearpage

:RIJ_YES_BEGIN:
\subsubsection{Rho and Partner Tensor, Wall Source}

Ensemble: :ENSEMBLE:

This is the $\gamma_3 \otimes \gamma_2 \gamma_3$ rho, parity partner $\gamma_1 \gamma_2 \otimes \gamma_1$ tensor, measured with a wall source.

{\small{Used :ACT_MEAS: measurements separated by :FREQ_MEAS: trajectories (:MISS_MEAS: missing). Blocksize of :RIJ_MEASPERBLOCK: measurements. Thermalization cut of trajectory :THERM:. Measurements performed up to trajectory :LAST_MEAS:. Prefered $t_{min}$ is :RIJ_TMIN:, consistent $t_{max}$ of :RIJ_TMAX: was used for all fits.}}


\begin{multicols}{2}
\begin{figure}[H]
\centering
\includegraphics[width=2.5in]{/projectnb/qcd/Staggered/:ENSEMBLE:/spectrum2/multifits/plots/multifits-rij.pdf}
\caption{Non-linear fit mass for the vector with partner $b_1$ tensor from a wall source.}
\end{figure}
\columnbreak
\begin{figure}[H]
\centering
\includegraphics[width=2.5in]{/projectnb/qcd/Staggered/:ENSEMBLE:/spectrum2/multifits/plots/multifits-rij-pv.pdf}
\caption{$p$ value for a non-linear fit of the vector with partner $b_1$ tensor from a wall source.}
\end{figure}
\end{multicols}

\begin{multicols}{2}
\begin{figure}[H]
\centering
\includegraphics[width=2.5in]{/projectnb/qcd/Staggered/:ENSEMBLE:/spectrum2/multifits/plots/multifits-rij-dir.pdf}
\caption{Zoom in on ground state mass of the vector from a wall source.}
\end{figure}
\columnbreak
\begin{figure}[H]
\centering
\includegraphics[width=2.5in]{/projectnb/qcd/Staggered/:ENSEMBLE:/spectrum2/multifits/plots/multifits-rij-osc.pdf}
\caption{Zoom in on ground state mass of the $b_1$ tensor from a wall source.}
\end{figure}
\end{multicols}

\begin{figure}[H]
\centering
\includegraphics[width=2.5in]{/projectnb/qcd/Staggered/:ENSEMBLE:/spectrum2/multifits/plots/multifits-rij-dir.pdf}
\caption{I should be a correlator plot!}
\end{figure}

\clearpage
:RIJ_YES_END:

:NU_YES_BEGIN:
\subsubsection{Nucleon in $\mathbf{8}$ Rep, Wall Source}

Ensemble: :ENSEMBLE:

This is the $\mathbf{8}$-rep Nucleon (positive and negative parity), measured with a wall source.

{\small{Used :ACT_MEAS: measurements separated by :FREQ_MEAS: trajectories (:MISS_MEAS: missing). Blocksize of :NU_MEASPERBLOCK: measurements. Thermalization cut of trajectory :THERM:. Measurements performed up to trajectory :LAST_MEAS:. Prefered $t_{min}$ is :NU_TMIN:, consistent $t_{max}$ of :NU_TMAX: was used for all fits.}}

\begin{multicols}{2}
\begin{figure}[H]
\centering
\includegraphics[width=2.5in]{/projectnb/qcd/Staggered/:ENSEMBLE:/spectrum2/multifits/plots/multifits-nu.pdf}
\caption{Non-linear fit mass for the $\mathbf{8}$-rep nucleon and its parity partner from a wall source.}
\end{figure}
\columnbreak
\begin{figure}[H]
\centering
\includegraphics[width=2.5in]{/projectnb/qcd/Staggered/:ENSEMBLE:/spectrum2/multifits/plots/multifits-nu-pv.pdf}
\caption{$p$ value for a non-linear fit of the $\mathbf{8}$-rep nucleon and its parity partner from a wall source.}
\end{figure}
\end{multicols}

\begin{multicols}{2}
\begin{figure}[H]
\centering
\includegraphics[width=2.5in]{/projectnb/qcd/Staggered/:ENSEMBLE:/spectrum2/multifits/plots/multifits-nu-dir.pdf}
\caption{Zoom in on ground state mass of the $\mathbf{8}$-rep nucleon from a wall source.}
\end{figure}
\columnbreak
\begin{figure}[H]
\centering
\includegraphics[width=2.5in]{/projectnb/qcd/Staggered/:ENSEMBLE:/spectrum2/multifits/plots/multifits-nu-osc.pdf}
\caption{Zoom in on ground state mass of the the $\mathbf{8}$-rep nucleon (negative parity) from a wall source.}
\end{figure}
\end{multicols}

\begin{figure}[H]
\centering
\includegraphics[width=2.5in]{/projectnb/qcd/Staggered/:ENSEMBLE:/spectrum2/multifits/plots/multifits-nu-dir.pdf}
\caption{I should be a correlator plot!}
\end{figure}

\clearpage
:NU_YES_END:

